%%%%%%%%%%%%%%%%%%%%%%%%%%%%%%%%%%%%%%%%%%%%%%%%%%%%%%%%%%%%%%%%%%%%%%%%%%%%%%%
% Chapter 3: Procedimiento experimental 
%%%%%%%%%%%%%%%%%%%%%%%%%%%%%%%%%%%%%%%%%%%%%%%%%%%%%%%%%%%%%%%%%%%%%%%%%%%%%%%

Este cap�tulo ha de contar con seccciones para la descripci�n de los experimentos 
y del material.
%
Tambi�n debe haber una secci�n para los resultados obtenidos y una �ltima de 
an�lisis de los resultados.

%++++++++++++++++++++++++++++++++++++++++++++++++++++++++++++++++++++++++++++++

\section{Programa en Pythom}
\label{3:sec:1}
\begin{verbatim}
#! encoding: UTF-8
#! /usr/bin/python 

import sin from math

Cero=0.00001

def f(x):
 return sin(x)

def biseccion(a,b,tol):
 c=float((a+b)/2.0)
 while f(c) != Cero) and abs(b-a) > tol:
  if f(a) * f(c) < Cero
   b = c;
  else
   a = c;
  c = (a+b)/2.0
 return c

print 'Calcular la raíz de sen de x'
a = float(raw_input('Valor a del intervalo: '))
b = float(raw_input('Valor b del intervalo: '))
t = 0.00000000000001
r = biseccion(a,b,t)
print "El valor de la raíz de seno de x es: %f"%(r)
\end{verbatim}


%++++++++++++++++++++++++++++++++++++++++++++++++++++++++++++++++++++++++++++++
\section{Resultados obtenidos}
\label{3:sec:2}

%------------------------------------------------------------------------------
\begin{figure}[!th]
\begin{center}
\includegraphics[width=0.75\textwidth]{images/figura1.eps}
\caption{Ejemplo de figura}
\label{fig:1}
\end{center}
\end{figure}
%------------------------------------------------------------------------------


%------------------------------------------------------------------------------
\input{tables/table.tex}
%------------------------------------------------------------------------------


