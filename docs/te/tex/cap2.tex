\section{Método de bisección}
\begin{enumerate}
 \item
   Se basa en el Teorema del Valor Intermedio (TVI), el cual establece que toda función continua f en un intervalo cerrado [a,b] toma todos los valores que se hallan entre f(a) y f(b). 
 \item
   Esto es que: todo valor entre f(a) y f(b) es la imagen de al menos un valor en el intervalo [a,b]. 
 \item
   En caso de que f(a) y f(b) tengan signos opuestos, el valor cero sería un valor intermedio entre f(a) y f(b), por lo que con certeza existe un p en [a,b] que cumple f(p) = 0. 
 \item
   De esta forma, se asegura la existencia de al menos una solución de la ecuación f(x) = 0.
\end{enumerate}
